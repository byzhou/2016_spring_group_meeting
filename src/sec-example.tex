%% -*- mode: latex; mode: auto-revert -*-
\maketitle

\begin{abstract}
  Basic document that demonstrates the {\LaTeX} build flow that I use as well as several quirks regarding specific packages.
\end{abstract}

\section{Example Section}
Claude Shannon wrote a good masters thesis while at \todo{Figure out where Shannon went to school...}some university~\cite{shannon1938}.
The thesis was really interesting\schuyler{What did he talk about?}.
Early work by George Boole laid the foundations of logic~\missingcitation.
Figure~\ref{fig-shannon} shows the realization of Boole's logical constructs in actual hardware.

\begin{figure}[h]
  \missingfigure{Add a figure that shows the mapping of logical constructs to circuit elements.
    Not sure how to do this, though.}
  \caption{}
  \label{fig-shannon}
\end{figure}

\section{Scripts}
I currently use one main script for keeping {\LaTeX} code in a \emph{nice} state for version control.
Specifically, I use a script, \texttt{fmtlatex}, to enforce a one-sentence-per-line rule on all files in the source directory that begin with \texttt{sec-}.
The included Makefile will, by default, run this on all source files that match this criteria.
I'm not dead-set that this is the right way to go about this, but it's how this is setup.
The Makefile can be modified, trivially, to use the \texttt{noformat-build} target instead of the \texttt{format-build} target which will not run \texttt{fmtlatex} on input source files.

\section{PGFPlots}
PGFPlots is an extremely versatile and useful, but has a ridiculously high learning curve.
It's highly advisable to use the \href{http://mirrors.ctan.org/graphics/pgf/contrib/pgfplots/doc/pgfplots.pdf}{PGFPlots manual} as a reference whenever designing these plots.
The colors that ship with PGFPlots are atrocious, however, and look terrible on a black and white printer.
I generally try to use a variety of colors, but to also use a variety of lightnesses to preserver some semblance of the original plots when printed on a terrible black and white printer.
\href{http://colorbrewer2.org/}{Colorbrewer} provides a good jumping off point for this for single hue colors.
In this directory, I include a forked submodule, \texttt{palette-art}, which generates a {\TeX} input file that defines all the colorbrewer colors.

\begin{figure}[h]
  %\centering \input{fig-example}
  \caption{Example PGFPlots figure}
\end{figure}
